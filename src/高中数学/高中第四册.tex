\documentclass[12pt,UTF8]{ctexbook}
\usepackage{ctex}
\usepackage{array}
\usepackage{graphicx}
\usepackage{wrapfig}
\usepackage[table,dvipsnames]{xcolor}
\usepackage{tabularx}
\usepackage{amsmath}
\usepackage{amssymb}
\usepackage{xfrac}
\usepackage{eucal}
\usepackage{titlesec}
\usepackage{amsthm}
\usepackage{tikz-cd}
\usepackage{enumitem}
\usepackage{verbatim}
\usepackage{fontspec,xunicode,xltxtra}
\usepackage{xeCJK} 

\definecolor{gl}{RGB}{246, 252, 240}
\definecolor{gd}{RGB}{236, 244, 230}
\definecolor{bg}{RGB}{242, 244, 228}


\setCJKmainfont[BoldFont=STZhongsong]{STSong}
\setCJKmonofont{simkai.ttf} % for \texttt
\setCJKsansfont{simfang.ttf} % for \textsf
\setlength\parskip{8pt}
\setlength{\fboxsep}{12pt}
\renewcommand\thesection{\arabic{chapter}.\arabic{section}}
\newtheorem{df}{定义}[section] 
\newtheorem{pp}{命题}[section]
\newtheorem{tm}{定理}[section]
\newtheorem{ex}{例子}[section]
\newtheorem{et}{例题}[section]
\newtheorem{sk}{思考}[section]
\newtheorem*{so}{解答}
\newenvironment{proof2}{\paragraph{\textbf{证明:}}}{\hfill$\square$}
\newtheorem{xt}{习题}[section]
\newtheorem{cor}{推论}[pp]
% 列举环境的行间距
\setenumerate[1]{itemsep=0pt,partopsep=0pt,parsep=0pt,topsep=0pt}
\setitemize[1]{itemsep=0pt,partopsep=0pt,parsep=0pt,topsep=0pt}
\setdescription{itemsep=0pt,partopsep=0pt,parsep=0pt,topsep=0pt}
% 章节字体大小
\titleformat{\section}{\zihao{-2}\bfseries}{ \thesection }{16pt}{}
% 封面
\title{\zihao{0} \bfseries 第四册}
\author{\zihao{2} \texttt{大青花鱼}}
% \date{\bfseries\today}
\date{}
% 正文
\begin{document}
\maketitle
\tableofcontents
\newpage

\chapter{平直空间}
% 简介平直空间(线性空间)。
% 直和
% 基底
% 维数

\section{平直空间的直映射}
% 将作为点映射的直映射扩展为一般空间到空间的映射(线性映射)。

\section{直映射的容与核}
% 介绍容(rank)与核(kernel),基本的
我们知道,直映射可以把直线映射为直线或点。那么,直映射把整个平面映射成什么呢?

前面已经说过,给定基底,直映射的性质由它在基底上的映射结果确定。
因此,我们可以通过研究直映射对基底的映射结果,来理解直映射对整个平面的映射结果。

给定基底$B$:$\mathbf{e}_1, \mathbf{e}_2$,考虑直映射$f$作用在基底向量上的结果:
$f(\mathbf{e}_1)$、$f(\mathbf{e}_2)$。平面的向量总是$\mathbf{e}_1$、$\mathbf{e}_2$的直组合,因此,
经过$f$映射后也是$f(\mathbf{e}_1)$、$f(\mathbf{e}_2)$的直组合。
$f(\mathbf{e}_1)$、$f(\mathbf{e}_2)$这两个向量是否共轴,决定了$f$的重要性质。

如果$f(\mathbf{e}_1)$、$f(\mathbf{e}_2)$不共轴,说明$f(\mathbf{e}_1)$、$f(\mathbf{e}_2)$也是平面的基底。
因此,它们的直组合就能生成平面任何向量。因此,$f$会把整个平面映射为平面。这时候我们说$f$是满射。

如果$f(\mathbf{e}_1)$、$f(\mathbf{e}_2)$都是零向量,那么它们的直组合总是零向量。
这说明无论什么向量,经过$f$映射都变成零向量。也就是说,$f$是零映射。

如果$f(\mathbf{e}_1)$、$f(\mathbf{e}_2)$共轴,但不都是零向量,那么它们的直组合总在一条过原点的直线上。
这说明无论什么向量,经过$f$映射后都在这条直线上。反之,这条直线上任一点都是某个向量经过$f$映射的结果。
也就是说,$f$把平面映射为一条(过原点的)直线。

\section{值映射}
% 把点映射到标量的映射(linear form)

\section{直映射的分解}
% 三维空间中直映射的分解

\end{document}