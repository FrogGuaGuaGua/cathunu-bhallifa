\documentclass[12pt,UTF8,a4paper]{article}
\usepackage{ctex}
\usepackage{array}
\usepackage{graphicx}
\usepackage{wrapfig}
\usepackage[table,dvipsnames]{xcolor}
\usepackage{tabularx}
\usepackage{longtable}
\usepackage{amsmath}
\usepackage{amssymb}
\usepackage{xfrac}
\usepackage{eucal}
\usepackage{titlesec}
\usepackage{amsthm}
\usepackage{tikz-cd}
\usepackage{enumitem}
\usepackage{verbatim}
\usepackage{fontspec,xunicode,xltxtra}
\usepackage{xeCJK} 
\usepackage{caption}
\usepackage{thmtools, thm-restate}
\usepackage{tcolorbox}	
\usepackage{geometry}
\usepackage{fancyhdr} %设置全文页眉、页脚的格式

\tcbuselibrary{breakable}
\tcbuselibrary{most}
% 修改脚注的编号为加圈样式,并且各页单独编号
\usepackage{pifont}
\usepackage[perpage,symbol*]{footmisc}
\DefineFNsymbols{circled}{{\ding{192}}{\ding{193}}{\ding{194}}
{\ding{195}}{\ding{196}}{\ding{197}}{\ding{198}}{\ding{199}}{\ding{200}}{\ding{201}}}
\setfnsymbol{circled}

\definecolor{gl}{RGB}{246, 252, 240}
\definecolor{gd}{RGB}{236, 244, 230}
\definecolor{bg}{RGB}{242, 244, 228}
\definecolor{neonblue}{RGB}{30, 90, 255}

\setCJKmainfont[BoldFont=STZhongsong, AutoFakeSlant=0.2]{STSong}
\setCJKmonofont{simkai.ttf} % for \texttt
\setCJKsansfont{simfang.ttf} % for \textsf
\setlength\lineskip{6pt}
\setlength\parskip{12pt}
\setlength{\fboxsep}{12pt}
% \renewcommand\thesection{\arabic{chapter}.\arabic{section}}
\newcommand{\arccot}{\operatorname{arccot}}
\newcommand{\dlim}[1]{^{\color{gray}\prime}#1}
\newcommand{\lian}[1]{
    \underset{#1}{\operatorname{lian}\,}
}
\newcommand{\di}[1]{\,\mathrm{d}#1}
% developpements limites
\newcommand{\oveq}[1]{\overset{#1}{=}} 
\newcommand{\olim}[1]{\mathit{o}\left(#1\right)}  % petit o
\newcommand{\Olim}[1]{\mathcal{O}\left(#1\right)}  % grand O
\newcommand{\Tlim}[1]{\mathcal{\Theta}\left(#1\right)}  % grand theta
\newcommand{\eqlim}[1]{\overset{#1}{\sim}}  % equivalence
\newcommand{\vect}[1]{\left\langle #1 \right\rangle}


\renewenvironment{proof}{\paragraph{\textbf{证明:}}}{\hfill$\square$}

\newtheorem{df}{定义}[section] 
\newtheorem{pp}{命题}[section]
\newtheorem{tm}{定理}[section]
\newtheorem{ex}{例子}[section]
\newtheorem{et}{例题}[section]
\newtheorem{sk}{思考}[section]
\newtheorem*{po}{公理}
\newtheorem*{so}{解答}
\newtheorem{xt}{习题}[section]
\newtheorem{cor}{推论}[pp]

\newtcolorbox{blockin}[2][]
  {colback = white, colframe = magenta!75!black, fonttitle = \bfseries,
    colbacktitle = magenta!85!black, enhanced,
    attach boxed title to top left={xshift=5mm, yshift=-2mm},breakable, 
    title=#2, #1}

\newtcolorbox{blockaft}[2][]
{colback = white, colframe = neonblue!105!black, fonttitle = \bfseries,
    colbacktitle = neonblue!125!black, enhanced,
    attach boxed title to top left={xshift=5mm, yshift=-2mm},breakable, 
    title=#2, #1}

% 列举环境的行间距
\setenumerate[1]{itemsep=0pt,partopsep=0pt,parsep=0pt,topsep=0pt}
\setitemize[1]{itemsep=0pt,partopsep=0pt,parsep=0pt,topsep=0pt}
\setdescription{itemsep=0pt,partopsep=0pt,parsep=0pt,topsep=0pt}
% 章节字体大小
\titleformat{\section}{\zihao{-2}\bfseries}{ \thesection }{16pt}{}
% 封面
\title{\zihao{0} \bfseries 思理入门}
\author{\zihao{2} \texttt{大青花鱼}}
% \date{\bfseries\today}
\date{}

% 正文
\begin{document}
\maketitle
% \tableofcontents
% \newpage
%%%%%%%%%%%%%%%%%%%%%%%%%%%%%%%%%%%%%%%%%%%%%%%%%%%%%%%%
\lhead{}% 页眉左边设为空
\chead{}% 页眉中间设为空
\rhead{}% 页眉右边设为空
\lfoot{}% 页脚左边设为空
\cfoot{\thepage}% 页脚中间显示页码
\rfoot{}% 页脚右边设为空
%%%%%%%%%%%%%%%%%%%%%%%%%%%%%%%%%%%%%%%%%%%%%%%%%%%%%%%%

人天生会思考,不过,不是每个人都能正确、有效地思考。有些人想问题很快,一下就能抓住重点;有些人想问题很慢,总是抓不住重点。
这是因为思考的方式不同。

在漫长的历史中,人类逐渐总结出一些思考的经验和方法。这些经验和方法对所有人适用,它们针对的不是思考的具体对象,而是思考的形式。
我们把这些经验和方法称为思理。下面就来谈谈其中最基本的内容。

\clearpage

\tableofcontents

\clearpage

\section{属性和概念}

世界上有种种事物,事物有各种各样的性质,事物之间有各种各样的联系。事物的性质和关系,都叫做\textbf{事物的属性}。

事物总有属性。属性相同的事物形成一类,属性不同的事物分别形成不同的类。

某类事物都有而其他类没有的属性,称为该类事物的\textbf{特性}。
其中,决定这类事物和其他事物不同的,称为\textbf{本性}。

我们通过看、听、闻、尝、触摸等方式认知事物。我们用感官从外部世界得到的感觉,形成了对单个事物的印象和记忆。
随着印象和记忆不断增多,我们会对它们进行归纳、总结、分类。

从印象和记忆中,我们可以进一步抽离出事物的属性,概括出一类事物的特有属性,并用语言来表记这类事物。
于是,通过使用语言,我们形成了反映事物的概念。\textbf{概念是反映事物的特有属性的思维形式}。

举例来说,人们在认识各种动物的时候,看见一只又一只兔子,形成了各种印象。
经过思考,人们认识到各种各样的兔子可以归为一类。

人们从对不同形状、大小、年龄、毛色的兔子的印象中,抽离出特有的属性,
认为有这样特性的动物属于一类,并创造专门的词“兔子”,来命名这类动物。这样,我们就形成了“兔子”的概念。

“兔子”的概念脱离了具体的印象。“兔子”并不就是具体的这只或那只兔子。我们说\textbf{概念是抽象的}。

“兔子”的概念建立后,就适用于所有的兔子。我们再看到新的兔子的时候,就会用“兔子”的概念来考虑它、理解它。
我们说\textbf{概念是普遍的}。

要注意的是:概念并不一定正确反映一类事物的真实属性。概念的特性只是我们主观概括得到的。
人类认识世界的能力是有局限的,很多概念都不一定正确反映了事物的真实属性。很多时候,我们对事物有了新的认识,
就会发现原有的认识是错误的,偏颇的,片面的。

很多时候,我们认识中的概念,客观世界里并不一定存在对应的事物。
比如:古人认为月亮上有月宫,海中有龙王,人死了变成鬼,等等。
“月宫”、“龙王”、“鬼”这些概念存在于人们的认知中,但并不正确反映客观世界的事物。

这种概念称为\textbf{虚假概念}。虚假概念可能来自错误的感知,错误的印象,错误的抽象,错误的思考,
错误的归纳概括,等等。

人类不断探索、研究世界,就是不断修正虚假概念,形成新概念的过程。

\begin{blockaft}{想一想}
    你觉得以下哪些是虚假概念?\\
    \begin{center}
        \begin{tabular}{p{8em}<{\centering} p{8em}<{\centering} p{12em}<{\centering} }
            孙悟空 & 黑洞 & 比$3$大比$4$小的自然数 \\
            & & \\
            最大的恒星 & 蓝色的苹果 & 一块谁也看不见的石头 \\
        \end{tabular}
    \end{center}
\end{blockaft}

人们使用\textbf{词语}来表记概念。\textbf{词语是概念的语言形式,概念是词语的思想内容}。

同一个概念可能用不同的词语表记,同一个词语也可能表记不同的概念。

日常语言不是完美的,语言使用中的混乱经常导致概念的混乱,而概念的混乱又会导致语言的混乱。
因此,我们希望创造更好的语言形式,减少概念的混乱,方便思考。

数学语言、计算机语言等,都是为了更好地表达和讨论复杂的概念而发明的语言。
我们把这样的语言称为人工语言,把人类历史中自然形成的语言称为自然语言。

\section{含义和范围}

概念的\textbf{含义},就是概念的特性,也叫概念的\textbf{内涵}。简单来说,就是“是什么”。

概念的\textbf{范围},就是具有概念特性的事物,也叫概念的\textbf{外延}。简单来说,就是“有哪些”。

比如:“人”这个概念的含义,就是人的特性,比如能够制造和使用生产工具、有智慧、能用语言交流、两足直立的灵长类动物。
而“人”的范围,就是张三、李四等一个个具体的人。
“资本主义国家”的含义,就是由资产阶级专政的国家。
而“资本主义国家”的范围,就是美国、日本、德国等一个个具体的资本主义国家。

有些概念的范围是独一无二的,称为\textbf{单独概念}或\textbf{个别概念}。
有些概念的外延是可以有很多的,称为\textbf{普遍概念}或\textbf{群体概念}。
比如,“李白”、“马克思”是单独概念,“国家”、“三角形”是群体概念。

\begin{blockaft}{想一想}
    以下哪些概念是单独概念,哪些是普遍概念?\\
    \begin{center}
        \begin{tabular}{p{8em}<{\centering} p{8em}<{\centering} p{8em}<{\centering} }
            工人 & 苹果 & 骆驼 \\
            & & \\
            张三 & 班长 & 一辆救护车\\
        \end{tabular}
    \end{center}
\end{blockaft}

\subsection{概念的关系}

% 等同/全同/重合/同一、从属/包含、互斥/全异、矛盾
不同的概念之间,有各种各样的关系。

两个概念:甲、乙的范围如果完全相同,就说甲\textbf{等同于}乙。

\begin{blockin}{例子}
    “《狂人日记》的作者”和“鲁迅”都指同一个人,范围完全相同。
    因此“《狂人日记》的作者”等同于“鲁迅”。
\end{blockin}

如果甲等同于乙,那么乙也等同于甲。反之亦然。因此也可以说甲乙\textbf{相等}或\textbf{相同}。
这种关系叫\textbf{等同关系}。

如果所有的甲都是乙,那么就说甲\textbf{从属于}乙,乙\textbf{包含}甲。
这种关系叫\textbf{从属关系}或\textbf{包含关系}。

\begin{blockin}{例子}
    所有的兔子都是动物,因此“兔子”从属于“动物”,“动物”包含“兔子”。 
\end{blockin}

如果有些甲是乙,也有一些甲不是乙,就说甲、乙\textbf{交叉}。
这种关系叫\textbf{交叉关系}。

\begin{blockin}{例子}
    有些面包是甜的,也有些面包不是甜的。于是“面包”和“甜的”是交叉关系。
\end{blockin}

如果任何甲都不是乙,任何乙也不是甲,就说甲、乙\textbf{全异},或甲、乙\textbf{互斥}。
这种关系叫\textbf{全异关系}或\textbf{互斥关系}。

\begin{blockin}{例子}
    任何兔子都不是荷花,任何荷花也不是兔子。我们说“兔子”和“荷花”全异或互斥。
\end{blockin}

全异关系中,有一种特殊的关系,叫做矛盾关系。如果甲、乙是互斥的概念,
并且都从属于另一个概念丙,而任何的丙要么是甲,要么是乙,就说甲、乙是矛盾的。
这种关系叫\textbf{矛盾关系}。

\begin{blockin}{例子}
    种子植物分为被子植物和裸子植物两类。“被子植物”和“裸子植物”都从属于“种子植物”,并且互斥。
    因此,“被子植物”和“裸子植物”是矛盾关系。
\end{blockin}

\begin{blockaft}{想一想}
    以下这些概念之间有什么关系?\\
    \begin{center}
        \begin{tabular}{p{8em}<{\centering} p{8em}<{\centering} p{8em}<{\centering} }
            灌木和乔木 & 汽车和公交车 & 妇女和工人 \\
            & & \\
            玉米和苞米 & 生铁和钢 & 豆干和豆皮 \\
        \end{tabular}
    \end{center}
\end{blockaft}

\subsection{定义}

定义是确定概念的含义的方法。

怎样定义呢?我们可以直接用自然语言描述概念的含义。

\begin{blockin}{例子}
姨妈就是妈妈的姐妹。 
\end{blockin}

这里要定义的概念是“姨妈”,“妈妈的姐妹”是概念的含义,“就是”这个词把两者等同起来。

\begin{blockin}{例子}
    月球就是地球唯一的卫星。 
\end{blockin}
    
这里要定义的概念是“月球”,“地球唯一的卫星”是概念的含义,“就是”这个词把两者等同起来。

此外,我们可以通过列举概念的范围来定义。

\begin{blockin}{例子}
    夏季大三角就是织女星、河鼓二和天津四。
\end{blockin}

这里要定义的概念是“夏季大三角”,“织女星、河鼓二和天津四”是概念的范围,“就是”把两者等同起来。

\begin{blockin}{例子}
    五行就是金、木、水、火、土。
\end{blockin}

这里要定义的概念是“五行”,“金、木、水、火、土”是概念的范围,“就是”把两者等同起来。

定义的方法有很多,还有一种常见的方法是“属加种差”。假设我们要定义甲,那么可以首先描述甲从属于某个更大的概念乙,
表示它是乙的一种,然后描述它和其他从属于乙的概念的差别。

\begin{blockin}{例子}
    偶数就是能被$2$整除的数。
\end{blockin}

这里要定义的概念是“偶数”,“偶数”从属于“数”,是“数”的一种,“能被$2$整除”描述了“偶数”和其他数的差别。

描述概念和其他事物的差别时,我们通常使用概念的属性,这些属性一般是概念的特性
,也可以是很多个接近特有的属性。

\begin{blockin}{例子}
    丹顶鹤是一种鹤,体长通常在$1.2$至$1.6$米之间,大多通体白色,头顶鲜红色,喉和颈黑色,耳至头枕白色,脚黑色。
\end{blockin}

这里要定义的概念是“丹顶鹤”。通过描述丹顶鹤的大小、形态特征来区别于其他的鹤。

\begin{blockaft}{想一想}
    找一找,以下这些概念是怎么定义的?使用了怎样的定义方法?\\
    \begin{center}
        \begin{tabular}{p{8em}<{\centering} p{8em}<{\centering} p{8em}<{\centering} }
            灌木 & 沸点 & 软钎焊 \\
            & & \\
            领事裁判权 & 乌托邦 & 分子生物学 \\
        \end{tabular}
    \end{center}
\end{blockaft}

\section{判断和命题}

\textbf{判断,是对事物、情况表达肯定或否定的态度}。我们在生活中说的每一句话,大多都是判断。
有些判断中还包含了其他的判断。这样的判断叫做\textbf{复合判断}。
如果判断中不包含其他的判断,就叫做\textbf{简单判断}。

\textbf{表示判断的语句叫做命题}。不严格地说,每个陈述句,都可以看作是命题。
简单判断对应的命题叫做\textbf{简单命题}。

表达肯定态度的简单判断(命题),称为肯定判断(命题);表达否定态度的简单判断(命题),称为否定判断(命题)。
否定命题一般带有否定词,比如“不”、“没有”、“并非”等等。

判断(命题)可以是真的,也可以是假的。是真是假,要看它是否符合客观世界的实际情况。

\begin{blockin}{例子}
    \begin{enumerate}
        \item “地球绕着太阳转”是真的。
        \item “鱼能飞”是假的。
    \end{enumerate}
\end{blockin}

我们约定,命题是可以讨论真假的语句。命题表示的判断,要么是真的,符合客观世界的真实情况,要么是假的。
到底是真是假,也许由于我们的认知能力有限,无法确定,但两者必居其一。不可能既真又假,也不可能非真非假。
不满足的语句,就不是命题。

我们可以发现,判断(命题)之间在真假方面是有联系的。如果掌握了其中的规律,
就可以通过已经知道是真的判断(命题),得知另一些判断(命题)的真假。我们把这种思考称为\textbf{推理}。

思理不是研究判断本身真假的学问,而是研究这些判断(命题)之间的联系,探索其中规律的学问。
不同的学科,研究的对象不同。但我们会发现,研究不同的对象时,所用到的思考规律和方法是共通的。
掌握了这些规律和方法,可以用在各种各样的地方。

因此,通过不断总结正确、有效的推理方式,我们就能更好更快地理解问题、分析问题、解决问题。

\begin{blockaft}{想一想}
    1. 假命题和虚假概念是一样的吗?\\
    2. 以下的命题是真的还是假的?\\
    2.1. 有的树叶是红的。 \\
    2.2. 他慢慢把窗帘拉上。\\
    2.3. 只要明天不刮台风,公园就照常开放。\\
    2.4. 他的姐姐比他高。
\end{blockaft}

\subsection{真值表}

为了方便、清楚地讨论命题之间的真假关系,我们常常使用叫做\textbf{真值表}的小工具,列举命题的真假。

一个判断(命题)的\textbf{真值},表示它到底是真的还是假的。
比如,我们说“地球绕着太阳转”的真值是“真”,“鱼能飞”的真值是“假”。

假设我们要研究一些命题之间的关系,比如从其中一些命题的真假出发,是否能推断出另一些命题的真假。
那么,我们可以列举作为出发点的命题的所有可能的真假情况,
然后在每一种情况下,调查另一些命题的真假,把这些真值排列成一个表,就是真值表。

\begin{blockin}{例子}
    用字母$p$表示命题:“地球绕着太阳转”,$q$表示命题:“地球不绕着太阳转”。

    我们想知道$p$的真假和$q$的关系。因此,我们从$p$出发,列举$p$所有可能的真假情况。
    考虑每一种情况下,$q$的真假。最后把结果排列成下表:
    
\begin{center}
    \begin{tabular}{ p{3em}<{\centering} p{3em}<{\centering}}
        \rowcolor{gd} $p$ & $q$ \\ [0.5ex] 
        \noalign{{\color{white}\hrule height 2pt}} % \hline\hline
        \rowcolor{gl} 真 & 假 \\   
        \noalign{{\color{white}\hrule height 2pt}}% \hline
        \rowcolor{gd} 假 & 真 \\
    \end{tabular}
\end{center}
\end{blockin}

从例子里的真值表可知,命题$p$的真假,总和$q$的真假相反。$p$为真,$q$就为假,反之亦然。


\begin{blockaft}{想一想}
    我们想要研究以下命题$p$和$q$的关系,请你画出真值表。\\
    1. $p$:有的树叶是红的。$q$:所有的树叶都是红的。 \\
    2. $p$:他不是小明的哥哥。$q$:他是小明的弟弟。\\
    3. $p$:天马上要亮了。$q$:天还没亮。\\
    4. $p$:他谁也不认识。$q$:他认识老张。
\end{blockaft}

\section{简单判断}

我们首先来看简单判断(命题)。
简单判断是不包含其他判断的判断。简单判断有两种,一种是\textbf{性质判断},一种是\textbf{关系判断}。

\subsection{性质判断和关系判断}

性质判断就是断定某个事物是否具有某种属性的判断。对应的命题称为\textbf{性质命题}。
性质判断一般只有一个概念作为主语,作为要判断的对象,以及一个谓语,作为要判断的内容。

\begin{blockin}{例子}
    鸟能飞。
    \begin{enumerate}
        \item 主语是“鸟”,就是要判断的对象;
        \item 谓语是“能飞”,就是要判断的内容。
    \end{enumerate}
\end{blockin}

关系判断就是断定事物之间是否具有某种关系的判断。对应的命题称为\textbf{关系命题}。
关系判断可以有多个主语或宾语,但只有一个述语,作为要判断的关系。
作为主语和宾语的概念都是关系涉及的对象,称为关系的\textbf{元}。

一个关系如果涉及两个元,就称为\textbf{二元关系},对应的命题称为二元关系命题;
如果涉及三个元,就称为三元关系,对应的命题称为三元关系命题,\textbf{依此类推}。

\begin{blockin}{例子}
    他比我高。
    \begin{enumerate}
        \item 要判断的关系是“比……高”,它涉及两个元,是二元关系;
        \item 两个元分别是“他”和“我”。
    \end{enumerate}
\end{blockin}

\subsection{全判断和有判断}

\begin{blockin}{例子}
    鸟能飞。\\
    所有鸟都能飞。\\
    有的鸟能飞。
\end{blockin}

观察这三句话,它们有什么不同?

思考“鸟能飞”的时候,我们可以给“鸟”这个概念加上两种描述。一种让我们思考“鸟”这个概念范围里的全体,
一种让我们思考这个概念范围里的部分。

“所有鸟都能飞”里,“所有鸟”表示“鸟”这个概念范围里的全体,我们称为“全称”。
讨论全称的判断(命题),叫做\textbf{全判断}(\textbf{全命题})。
指示全称的修饰语有:“所有”、“全部”、“任何”、“一切”、“每个”、“总”、“都”等。

“所有鸟都能飞”里,“有的鸟”表示“鸟”这个概念范围里的全体,我们称为“有称”。
讨论有称的判断(命题),叫做\textbf{有判断}(\textbf{有命题})。
指示有称的修饰语有:“有的”、“有些”、“某些”、“存在”、“至少有一个”等。

要注意的是,有称也可能指概念范围的全体。比如,某个学校一共有六个年级,
当我们说“有的年级今天放假”的时候,可能有$1$、$2$、$3$、$4$、$5$、$6$个年级今天放假,
其中就包括了“所有六个年级今天都放假”的情形。

当概念的范围是独一无二(也就是单独概念)的时候,有称和全称没区别,我们称为“单称”,
把讨论它的判断(命题)叫做\textbf{单判断}(\textbf{单命题})。

全判断、有判断、单判断,每个都可以表达肯定和否定的态度,因此性质判断一共六类:
全肯定判断、全否定判断、有肯定判断、有否定判断、有肯定判断、有否定判断。对应的命题分别叫:
全肯定命题、全否定命题、有肯定命题、有否定命题、有肯定命题、有否定命题。

\begin{blockaft}{想一想}
    说一说,以下的简单命题是哪种命题?\\
    1. 有的树叶是红的。 \\
    2. 所有的树叶都是红的。\\
    3. 他不是小明的哥哥。\\
    4. 我有两个足球。 \\
    5. 无法在陆地生活的哺乳动物,也是有的。\\
    6. 他至少有五十岁了。\\
    7. 没有人能从那个摊主手里赢钱。 \\
    8. 他谁也不认识。
\end{blockaft}

\subsection{简单命题的反命题}

简单命题的\textbf{反命题},是形式上表达的态度和它相反的命题。

性质命题的反命题,就是否定性质判断的谓语。关系命题的反命题,就是否定关系判断的述语。

单肯定命题的反命题,一般就是它对应的否定句;单否定命题的反命题,一般就是它对应的肯定句。

\begin{blockin}{例子}
    \begin{enumerate}
        \item “小明是小红的哥哥”的反命题是“小明不是小红的哥哥”。
        \item “他没有读过《三国演义》”的反命题是“他读过《三国演义》”。
        \item “他每天至少刷两次牙”的反命题是“他每天至多刷一次牙”。
        \item “他比我高”的反命题是“他不比我高”。
        \item “有些蛇不吃青蛙”的反命题是“有些蛇吃青蛙”。
    \end{enumerate}
\end{blockin}

性质命题的主语可以是全称和有称,它的反命题也可以分别是全称和有称。我们把它们分别称为\textbf{全反命题}和\textbf{有反命题}。

全命题可以对应全反命题和有反命题。

\begin{blockin}{例子}
    “所有的鱼都能飞”对应的全反命题是“任何鱼都不能飞”,有反命题是“有的鱼不能飞”。
\end{blockin}

同样地,有命题也可以对应全反命题和有反命题。

\begin{blockin}{例子}
    “有的鱼能飞”对应的全反命题是“任何鱼都不能飞”,有反命题是“有的鱼不能飞”。
\end{blockin}

关系命题的每个元都可以是全称和有称,因此,二元关系命题就对应四个命题和四个反命题,三元关系命题对应八个,依此类推。
我们现在不研究关系命题的反命题。

假设性质命题的内容相同,只有全称、有称和肯定否定的部分不同,
对应的全命题、有命题、全反命题、有反命题之间有什么联系呢?已知其中一个的真假,能不能推断其他命题的真假?

\begin{blockaft}{想一想}
    请写出以下命题的反命题。\\
    1. 有的树叶是绿的。 \\
    2. 所有的树叶都是红的。\\
    3. 他不是小明的哥哥。\\
    4. 他家至少有两个孩子。 \\
    5. 这种产品是否有防皱纹的效果,还有待研究。\\
    6. 过了这座山,就能看到五彩湖了。\\
    7. 天马上要亮了。 \\
    8. 他谁也不认识。
\end{blockaft}

\subsection{简单命题的否定}

简单命题的\textbf{否定},是一个真假和原来的命题恰好相反的命题。
只要原来的命题是真的,它就是假的。
只要原来的命题是假的,它就是真的。

比如,命题“他至多有三支笔”的否定是“他至少有四只笔”。

对自然语言中具体的命题来说,它的否定可能有多种形式,和语言的种类、文体、措辞都有关系。
数学上,如果把某个命题记作$p$,那么为了方便,把$p$的否定记为“非$p$”。

\textbf{命题的否定是相互的},假设命题$p$是$q$的否定,那么$q$也是$p$的否定。

\begin{blockaft}{想一想}
    相互矛盾的概念,和相互否定的命题,有什么相似之处,有什么不同之处,有什么联系?
\end{blockaft}

对简单命题来说,命题的否定取决于全称、有称还是单称。现在我们只讨论性质命题的否定,不讨论关系命题的否定。

\textbf{单命题的否定就是它的反命题}。

\begin{blockin}{例子}
    \begin{enumerate}
        \item “小明是小红的哥哥”的否定是“小明不是小红的哥哥”。
        \item “他没有读过《三国演义》”的否定是“他读过《三国演义》”。
        \item “他每天至少刷两次牙”的否定是“他每天至多刷一次牙”。
    \end{enumerate}
\end{blockin}

全命题和有命题的否定,则要复杂一点。

\begin{blockin}{例子}
    用字母$p$表示命题:“四年级所有的学生都参加了校运会”,$q$表示命题:“四年级有些学生没有参加校运会”
    ,$r$表示命题:“四年级所有的学生都没有参加校运会”。

    考虑$p$的真假和$q$、$r$的关系,结果为下表:
    
\begin{center}
    \begin{tabular}{ p{3em}<{\centering} p{3em}<{\centering} p{3em}<{\centering}}
        \rowcolor{gd} $p$ & $q$ & $r$ \\ [0.5ex] 
        \noalign{{\color{white}\hrule height 2pt}} % \hline\hline
        \rowcolor{gl} 真 & 假 & 假 \\   
        \noalign{{\color{white}\hrule height 2pt}}% \hline
        \rowcolor{gd} 假 & 真 & -- \\
    \end{tabular}
\end{center}
\end{blockin}

我们发现,当“四年级所有的学生都参加了校运会”为真的时候,“四年级所有的学生都没有参加校运会”为假;
但“四年级所有的学生都参加了校运会”为假的时候,“四年级所有的学生都没有参加校运会”的真假无法判断。
比如,有可能四年级大部分学生参加了校运会,但少数学生没有参加,这时候“四年级所有的学生都没有参加校运会”为假。
但也有可能四年级确实没有任何学生参加校运会。

不过,我们也可以看到,“四年级所有的学生都参加了校运会”和“四年级有些学生没有参加校运会”的真假恰好相反。

一般来说,\textbf{全命题的否定是它对应的有反命题}。比如,“所有的甲是乙”的否定,是“有些甲不是乙”;
“所有的甲都不是乙”的否定,是“有些甲是乙”。

另一方面,\textbf{有命题的否定是它对应的全反命题}。比如,“有些甲是乙”的否定,是“所有的甲都不是乙”;
“有些甲不是乙”的否定,是“所有的甲都是乙”。

\begin{blockin}{例子}
    “有的树秋天会落叶”的否定是“所有的树秋天都不会落叶”,“有的树秋天不会落叶”的否定是“所有的树秋天都会落叶”。
\end{blockin}

\begin{blockaft}{想一想}
    请写出以下命题的否定。\\
    1. 有的树叶是红的。\\
    2. 所有的树叶都是红的。 \\
    3. 他不是小明的哥哥。\\
    4. 有些蚊子不咬人。\\
    5. 无法在陆地生活的哺乳动物,也是有的。\\
    6. 没人知道他去了哪里。
\end{blockaft}

\subsection{二元关系}

关系判断涉及的概念比性质判断多,因此更加复杂。这里我们只讨论最简单的二元关系。
如果把二元关系记作$R$,两个元记为“甲”、“乙”,那么二元关系命题可以记作“甲$R$乙”。

事物之间的关系有很多种。关系也有各种性质。

如果甲$R$乙和乙$R$甲同时为真,同时为假,就说$R$是对称的,叫做\textbf{对称关系}。
如果只有甲就是乙的时候,甲$R$乙和乙$R$甲才同时为真,否则甲$R$乙和乙$R$甲不会同时为真,就说$R$是反称的,叫做\textbf{反称关系}。
如果甲$R$乙和乙$R$甲总是一真一假,就说$R$是逆称的,叫做\textbf{逆称关系}。

\begin{blockin}{例子}
    “等于”是一种对称关系。甲等于乙,那么乙也等于甲。反之亦然。
    “大于等于”是一种反称关系。只有甲等于乙的时候,才有甲大于等于乙并且乙大于等于甲。   
    “大于”是一种逆称关系。甲大于乙,那么乙必定不大于甲。反之亦然。
\end{blockin}

如果甲$R$甲总是真的,就说$R$是自返的,是\textbf{自返关系}。

\begin{blockin}{例子}
    “等于”是一种自返关系。甲等于甲总是真的。
\end{blockin}

如果任何情况下,甲$R$乙和乙$R$甲至少有一个是真的,就说$R$是完全的,叫做\textbf{完全关系}。

\begin{blockin}{例子}
    自然数的“大于等于”是一种完全关系。任取两个自然数甲和乙,要么甲大于等于乙,要么乙大于等于甲。
\end{blockin}

如果甲$R$乙和乙$R$丙都是真的时候,甲$R$丙也总是真的,就说$R$是传递的,是\textbf{传递关系}。

\begin{blockin}{例子}
    “等于”和“大于”都是传递关系。
    甲等于乙、乙等于丙的时候,甲也等于丙。
    甲大于乙、乙大于丙的时候,甲也大于丙。
\end{blockin}

同时满足对称、自返、传递的二元关系,称为\textbf{等价关系}。比如“等于”就是等价关系。

同时满足反称、自返、传递的二元关系,称为\textbf{偏序关系}。比如“大于等于”就是偏序关系。

既是偏序关系,又是完全关系,叫做\textbf{全序关系}。比如自然数的“大于等于”就是偏序关系。

\begin{blockaft}{想一想}
    1. 你还知道哪些对称关系、反称关系?\\
    2. 你还知道哪些自返关系? \\
    3. 你还知道哪些传递关系?\\
    4. 你还知道哪些等价关系?\\
    4. 你还知道哪些偏序关系?
\end{blockaft}

\section{复合判断}

复合判断(命题)就是包含了不止一个判断的判断(命题)。
复合判断的真假,取决于它包含的判断的真假。

常见的复合判断有\textbf{假言判断}、\textbf{或言判断}、\textbf{选言判断}和\textbf{联言判断}。


\subsection{联言判断}
联言判断是关于多个判断的全判断,它表示多个判断全是真的。对应的命题叫做联言命题。

\begin{blockin}{例子}
    这辆车不仅动力充足,外观也大气。
\end{blockin}

这句话提到了两个判断:“这辆车动力充足”和“这辆车外观大气”,它表达的意思是:这两个判断都是真的。
汉语里,联言命题一般使用“既……又……”,“不仅……而且……”这样表示并列关系的连词,把各个判断连起来。
每个判断称为它的\textbf{分支判断}。

联言判断的真值表为:
\begin{center}
    \begin{tabular}{ p{3em}<{\centering} p{3em}<{\centering} p{8em}<{\centering} }
        \rowcolor{gd} 甲 & 乙 & 甲,并且乙 \\ [0.5ex] 
        \noalign{{\color{white}\hrule height 2pt}} % \hline\hline
        \rowcolor{gl} 真 & 真 & 真  \\  
        \noalign{{\color{white}\hrule height 2pt}}% \hline
        \rowcolor{gd} 真 & 假 & 假  \\
        \noalign{{\color{white}\hrule height 2pt}}% \hline
        \rowcolor{gl} 假 & 真 & 假 \\  
        \noalign{{\color{white}\hrule height 2pt}}% \hline
        \rowcolor{gd} 假 & 假 & 假 \\
    \end{tabular}
\end{center}

\subsection{或言判断}
或言判断是关于多个判断的有判断,它表示多个判断里至少有一个是真的。对应的命题叫做或言命题。

\begin{blockin}{例子}
    或许是你说错了,或许是我听错了。
\end{blockin}

这句话提到了两个判断:“你说错了”和“我听错了”,它表达的意思是:这两个判断至少有一个是真的。
汉语里,或言命题一般使用“也许……也许……”,“……或者……”这样表示并列关系的连词,把各个判断连起来。
每个判断称为它的分支判断。

或言判断的真值表为:
\begin{center}
    \begin{tabular}{ p{3em}<{\centering} p{3em}<{\centering} p{8em}<{\centering} }
        \rowcolor{gd} 甲 & 乙 & 甲,或者乙 \\ [0.5ex] 
        \noalign{{\color{white}\hrule height 2pt}} % \hline\hline
        \rowcolor{gl} 真 & 真 & 真  \\  
        \noalign{{\color{white}\hrule height 2pt}}% \hline
        \rowcolor{gd} 真 & 假 & 真  \\
        \noalign{{\color{white}\hrule height 2pt}}% \hline
        \rowcolor{gl} 假 & 真 & 真 \\  
        \noalign{{\color{white}\hrule height 2pt}}% \hline
        \rowcolor{gd} 假 & 假 & 假 \\
    \end{tabular}
\end{center}

\subsection{选言判断}
选言判断也是关于多个判断的判断,它表示多个判断里恰有一个是真的。对应的命题叫做选言命题。

\begin{blockin}{例子}
    要么是你不对,要么是我不对。
\end{blockin}

这句话提到了两个判断:“你不对”和“我不对”,它表达的意思是:这两个判断恰有一个是真的。
汉语里,选言命题一般使用“要么……要么……”,“或者……或者……”这样表示并列关系的连词,把各个判断连起来。
每个判断称为它的分支判断。为了强调只有一个是真的,经常加上“两者不可得兼”、“两者必居其一”等说法。

选言判断的真值表为:
\begin{center}
    \begin{tabular}{ p{3em}<{\centering} p{3em}<{\centering} p{8em}<{\centering} }
        \rowcolor{gd} 甲 & 乙 & 要么甲,要么乙 \\ [0.5ex] 
        \noalign{{\color{white}\hrule height 2pt}} % \hline\hline
        \rowcolor{gl} 真 & 真 & 假  \\  
        \noalign{{\color{white}\hrule height 2pt}}% \hline
        \rowcolor{gd} 真 & 假 & 真  \\
        \noalign{{\color{white}\hrule height 2pt}}% \hline
        \rowcolor{gl} 假 & 真 & 真 \\  
        \noalign{{\color{white}\hrule height 2pt}}% \hline
        \rowcolor{gd} 假 & 假 & 假 \\
    \end{tabular}
\end{center}

\subsection{假言判断}
假言判断是关于条件的判断,它涉及两个称为\textbf{前件}和\textbf{后件}的判断,并判断前件是后件成立的条件。
假言判断并不对前件或后件做判断,只对两者的条件关系做判断,而条件关系又分为两种:\textbf{充分条件}和\textbf{必要条件}。

充分条件假言判断想说的是:前件为真的时候,后件也为真。

\begin{blockin}{例子}
    人只要不吃东西,就会饿。
\end{blockin}

这个例子里,前件是“人不吃东西”,后件是“人会饿”。它表达的意思是:前件真的时候,后件一定也是真的。
我们说,前件是后件的充分条件。这个例子里,“人不吃东西”是“人会饿”的充分条件。
汉语里,我们一般使用“如果……那么……”,“只要……就……”等连词表达充分条件假言判断。

\begin{blockaft}    
    读这句话:“你要是考了满分,太阳要从西边升上来了!”\\
    1. 这句话里,“你考了满分”和“太阳要从西边升上来”是什么关系?\\
    2. “你”如果没考满分,这个关系还成立吗?这时“太阳要从西边升上来”和“太阳不会从西边升上来”对这个判断有什么影响?
\end{blockaft}

充分条件判断的真值表为:
\begin{center}
    \begin{tabular}{ p{3em}<{\centering} p{3em}<{\centering} p{8em}<{\centering} }
        \rowcolor{gd} 甲 & 乙 & 如果甲,那么乙 \\ [0.5ex] 
        \noalign{{\color{white}\hrule height 2pt}} % \hline\hline
        \rowcolor{gl} 真 & 真 & 真  \\  
        \noalign{{\color{white}\hrule height 2pt}}% \hline
        \rowcolor{gd} 真 & 假 & 假  \\
        \noalign{{\color{white}\hrule height 2pt}}% \hline
        \rowcolor{gl} 假 & 真 & 真 \\  
        \noalign{{\color{white}\hrule height 2pt}}% \hline
        \rowcolor{gd} 假 & 假 & 真 \\
    \end{tabular}
\end{center}

必要条件假言判断想说的是:前件为假的时候,后件也为假。

\begin{blockin}
    人只有识字了,才能读书。
\end{blockin}

这个例子里,前件是“人识字”,后件是“人能读书”。不识字的人,不能读书。
前件假的时候,后件也是假的,这说明必要条件关系是真的。
我们说,前件是后件的必要条件。这个例子里,“识字”是“能读书”的必要条件。
汉语里,我们一般使用“只有……才……”,“仅当……时,有……”等连词表达充分条件假言判断。

\begin{blockaft}
    读这句话:“只有到了晚上,才能看见满天的星星。”\\
    1. 这句话里,“到了晚上”和“看见星星”是什么关系?\\
    2. 有时,到了晚上也看不见星星。是否说明这个判断不成立呢?
\end{blockaft}

必要条件判断的真值表为:
\begin{center}
    \begin{tabular}{ p{3em}<{\centering} p{3em}<{\centering} p{8em}<{\centering} }
        \rowcolor{gd} 甲 & 乙 & 只有甲,才有乙 \\ [0.5ex] 
        \noalign{{\color{white}\hrule height 2pt}} % \hline\hline
        \rowcolor{gl} 真 & 真 & 真  \\  
        \noalign{{\color{white}\hrule height 2pt}}% \hline
        \rowcolor{gd} 真 & 假 & 真  \\
        \noalign{{\color{white}\hrule height 2pt}}% \hline
        \rowcolor{gl} 假 & 真 & 假 \\  
        \noalign{{\color{white}\hrule height 2pt}}% \hline
        \rowcolor{gd} 假 & 假 & 真 \\
    \end{tabular}
\end{center}

如果前件既是后件的充分条件,也是后件的必要条件,就说前件是后件的\textbf{充要条件}。

充要条件一般用“当且仅当”句式来表达。比如:“李四年纪比张三大,当且仅当张三年纪比李四小。”

充要条件假言判断可以看成充分条件假言判断和必要条件假言判断的联言判断。因此,充要条件判断的真值,
可以通过充分条件判断的真值和必要条件判断的真值得到:
\begin{center}
    \begin{tabular}{ p{3em}<{\centering} p{3em}<{\centering} p{7em}<{\centering} p{7em}<{\centering} p{7em}<{\centering} }
        \rowcolor{gd} 甲 & 乙 & 如果甲,那么乙 & 只有甲,才有乙 & 甲,当且仅当乙 \\ [0.5ex] 
        \noalign{{\color{white}\hrule height 2pt}} % \hline\hline
        \rowcolor{gl} 真 & 真 & 真 & 真 & 真 \\  
        \noalign{{\color{white}\hrule height 2pt}}% \hline
        \rowcolor{gd} 真 & 假 & 假 & 真 & 假 \\
        \noalign{{\color{white}\hrule height 2pt}}% \hline
        \rowcolor{gl} 假 & 真 & 真 & 假 & 假 \\  
        \noalign{{\color{white}\hrule height 2pt}}% \hline
        \rowcolor{gd} 假 & 假 & 真 & 真 & 真\\
    \end{tabular}
\end{center}
可以看到,前件和后件都为真、都为假的时候,充要条件判断为真。前件和后件一真一假的时候,充要条件判断为假。

\subsection{用复合判断推理}


\end{document}